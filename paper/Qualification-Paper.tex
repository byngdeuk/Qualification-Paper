\documentclass[]{article}
\usepackage{lmodern}
\usepackage{amssymb,amsmath}
\usepackage{ifxetex,ifluatex}
\usepackage{fixltx2e} % provides \textsubscript
\ifnum 0\ifxetex 1\fi\ifluatex 1\fi=0 % if pdftex
  \usepackage[T1]{fontenc}
  \usepackage[utf8]{inputenc}
\else % if luatex or xelatex
  \ifxetex
    \usepackage{mathspec}
  \else
    \usepackage{fontspec}
  \fi
  \defaultfontfeatures{Ligatures=TeX,Scale=MatchLowercase}
\fi
% use upquote if available, for straight quotes in verbatim environments
\IfFileExists{upquote.sty}{\usepackage{upquote}}{}
% use microtype if available
\IfFileExists{microtype.sty}{%
\usepackage{microtype}
\UseMicrotypeSet[protrusion]{basicmath} % disable protrusion for tt fonts
}{}
\usepackage[margin=1in]{geometry}
\usepackage{hyperref}
\hypersetup{unicode=true,
            pdftitle={Diffusion of Laws Against Women Related Violence},
            pdfauthor={Byung-Deuk Woo},
            pdfborder={0 0 0},
            breaklinks=true}
\urlstyle{same}  % don't use monospace font for urls
\usepackage{graphicx,grffile}
\makeatletter
\def\maxwidth{\ifdim\Gin@nat@width>\linewidth\linewidth\else\Gin@nat@width\fi}
\def\maxheight{\ifdim\Gin@nat@height>\textheight\textheight\else\Gin@nat@height\fi}
\makeatother
% Scale images if necessary, so that they will not overflow the page
% margins by default, and it is still possible to overwrite the defaults
% using explicit options in \includegraphics[width, height, ...]{}
\setkeys{Gin}{width=\maxwidth,height=\maxheight,keepaspectratio}
\IfFileExists{parskip.sty}{%
\usepackage{parskip}
}{% else
\setlength{\parindent}{0pt}
\setlength{\parskip}{6pt plus 2pt minus 1pt}
}
\setlength{\emergencystretch}{3em}  % prevent overfull lines
\providecommand{\tightlist}{%
  \setlength{\itemsep}{0pt}\setlength{\parskip}{0pt}}
\setcounter{secnumdepth}{0}
% Redefines (sub)paragraphs to behave more like sections
\ifx\paragraph\undefined\else
\let\oldparagraph\paragraph
\renewcommand{\paragraph}[1]{\oldparagraph{#1}\mbox{}}
\fi
\ifx\subparagraph\undefined\else
\let\oldsubparagraph\subparagraph
\renewcommand{\subparagraph}[1]{\oldsubparagraph{#1}\mbox{}}
\fi

%%% Use protect on footnotes to avoid problems with footnotes in titles
\let\rmarkdownfootnote\footnote%
\def\footnote{\protect\rmarkdownfootnote}

%%% Change title format to be more compact
\usepackage{titling}

% Create subtitle command for use in maketitle
\newcommand{\subtitle}[1]{
  \posttitle{
    \begin{center}\large#1\end{center}
    }
}

\setlength{\droptitle}{-2em}

  \title{Diffusion of Laws Against Women Related Violence}
    \pretitle{\vspace{\droptitle}\centering\huge}
  \posttitle{\par}
    \author{Byung-Deuk Woo}
    \preauthor{\centering\large\emph}
  \postauthor{\par}
    \date{}
    \predate{}\postdate{}
  

\begin{document}
\maketitle

\subsubsection{Research Design}\label{research-design}

Diffusion of laws against women crimes in the world.

\subsection{Descriptive statistics for women
crimes}\label{descriptive-statistics-for-women-crimes}

\url{http://www.unwomen.org/en/what-we-do/ending-violence-against-women/facts-and-figures\#notes}
Show how severe the crimes are.

\subsection{Theory: When and Why Do Countries Adopt Laws Against Women
Crimes?}\label{theory-when-and-why-do-countries-adopt-laws-against-women-crimes}

\subsection{Data \& Method}\label{data-method}

Figure: Proliferation of Laws against Women Crimes (by sort of crimes)
Figure: Cumulative graph of the number of countries having laws against
women crimes. Figure: Survival Functions Map: Countries with Laws
against Women Crimes (by sort of crimes)

\subsection{Variables}\label{variables}

\section{Dependent Variables}\label{dependent-variables}

Adoption of Laws Against Women Crimes -
(evaw-global-database.unwomen.org) Description: I use brand-new database
for laws against women ccrimes from UN. UN WOMEN Global Database on
Violence against Women provides database about laws against women crimes
for 195 countries in the world. Developed in 2016, the database deals
with the ways how countries address violence against women: laws,
policies, budgets, services, prevention, perpetrators programmes,
regional and international initiatives, monitoring and evaluation, etc.
I will focus on the laws agasint women crimes because laws are the basic
methods for preventing crimes. Again, the database subcategorizes laws
into Constitutional Provision, Legislation, and Regulations. I use the
data related to Legislation because of several reasons. First,
Constitutional Provision is really hard to be changed. It means that
there are less variations. Second, Regulations are usually based on
legislation and less efficient than legislation. Therefore, I will focus
on Legislation.

(Citation: I have to re-write) The United Nations defines violence
against women as ``any act of gender-based violence that results in, or
is likely to result in, physical, sexual or psychological harm or
suffering to women, including threats of such acts, coercion or
arbitrary deprivation of liberty, whether occurring in public or in
private life'' (General Assembly Resolution 48/104 Declaration on the
Elimination of Violence against women, 1993).

Throughout the site, unless specified differently, the term ``women''
refers to females of all ages, including girls (UN General Assembly,
2006)

The sources of information of UN women global database on violence
against women: Government reports to human rights bodies, including the
Committee on the Elimination of Discrimination against Women (CEDAW) and
the Universal Periodic Review (UPR). Information provided by Governments
in follow-up to the Fourth World Conference on Women in Beijing (1995),
including: the outcomes of the twenty-third special session of the UN
General Assembly (2000); and the national review of implementation of
the Beijing Declaration and Platform for Action in the context of the
20th anniversary of the adoption of the Beijing Declaration and Platform
for Action (2015). Information available through official reports of
other relevant United Nations entities. Government statements made at
the United Nations.

\section{Explanatory Variables}\label{explanatory-variables}

Domestic Explanatory Variables GDP per capita Urbanization Regime Type
(Level of Democracy) Colonial Heritage Religion Geo-Political Region IO
Membership Domestic Violence Women legislators

\section{Diffusion Variables}\label{diffusion-variables}

Geographical contiguity

\section{Control Variables}\label{control-variables}

\section{Event History Analysis}\label{event-history-analysis}

Figure: Estimated Survival Probabilities of Law against women crimes for
195 countries Figure: Estimated Survival Probabilities of Law against
women crimes for 195 countries by Regime Type

\section{Result}\label{result}

\section{Conclusion}\label{conclusion}


\end{document}
